% Abstract for the TUM report document
% Included by MAIN.TEX


\clearemptydoublepage
\phantomsection
\addcontentsline{toc}{chapter}{Abstract}	





\vspace*{2cm}
\begin{center}
{\Large \bf Abstract}
\end{center}
\vspace{1cm}

An abstracts abstracts the thesis!

The monitoring of fish for stock assessment in aquaculture, commercial fisheries and in the assessment of the effectiveness of biodiversity management strategies such as marine protected Areas and closed area management has been thriving since the 1980s.
as does area constinuously grows, 
it becomes important to develop a remote monitoring system to estimate the biomass of the large number of fishes bred in cages,
since around 80\% of all sales of farmed fish ara arranged pre-harves,
that mean, the profit on the sale directly depends on correct estimations of weight,
size distribution and total biomass.
Therefore automated and relatively affordable tools for biomass estimation have to be developed.

Here, we will rely on complex stereo camera system,
compose of time of flight range camera and CCD grayscale camera,
that film fishes in the cage for sertain period of time.
in order to estimate the biomass, the volume of the fish has to be estimated. this can be achieved by first detecting and segmenting the fish in every grayscale image of the incoming video stream and then translate this found fish contour  to the range image obtaining a estimation of the volume.
to find the algorithm that is in line with our problem,
we need to understand the challenge in detecting fishes.
they include the motion of the fish which makes the object of interest deformable,
the location of the fish respect to the camera and occlussion caused by having multiple fishes in every available frame.

In this project, we concentrate on the first step that is detection of the fish that undergo deformation in grayscales images.
Inspired by recent works in \todo{ define recent work} 
we develop a similar approach for fish detection. we use a \todo { define algorithm}

We evaluate the proposes method by computing difference between tha label dataset with the predicted result,
in addition, we cluster the results from different camera locations and found that when the sagittal plane is parallel to the image plane,
the tracking algorithm provide the best result.
Finally, we show that \todo { show improvement}.

Therefore, this thesis accomplished the following:
\todo {define accomplish, if there are somethings}
% \listoftodos{ define papers}
% \missingfigure{}