\chapter{Conclusion}
\label{chapter:Conclusion}


\section{Conclusion}
In this thesis, we accomplished  the goals of creating a annotated datasets that
comprise of learning keypoints and fish contours from a set of 2D grayscale images.
We also implement a combine solution from a part based detection model and a template
matching detection approach, which is capable of predict keypoints and fish contours in an 
unlabeled 2d grayscale image and verifying the validity of the prediction.


\todo{more detail about the algorithm}
Inspired on the works from \citet{Ramanan2012} and \citet{Hinterstoisser2012}, we 
use the part based model
Based on the results, we conclude that the algorithm work best when the sagittal plane
of the fish is parallel to the image plane. In addition, we also conclude \todo{.....}

Lastly, we demonstrate our algorithm on multiple fish detection by \todo{......}

\section{Future Work}
Multiple fish detection and occlusions are some of the biggest challenges in this
project and needs to be addressed because they are always present in real situations.
Another important task for the future would be to increase the size of the labeled database
to improve the detection result, and perform a benchmarking of other approaches to
have a better picture of what are the challenges involve in this problem.

Since the final product combine a Time of flight camera and a 2D HD grayscale camera in a
stereo system rig inside a underwater housing, the next step would be use the strength
from each device into fusing the depth information with the 2D intensity image.
This present new challenges as the system is design with underwater purpose. the new approach
should deal with the systematical error introduce in perspective pinhole camera mode by several
refraction of the light rays causes by the housing configuration, due to the glass interface
between water and air or other inert gas introduce in the housing, this issue could be
handle by the approach propose in this work \citet{Sedlazeck2011}.

After accomplish a proper data fusion between the two modalities, the next step
would be find a algorithm capable of take advantage of all this information in a 
optimize way.

