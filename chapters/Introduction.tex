\chapter{Introduction}
\label{chapter:Introduction}



Here starts the thesis with an introduction. Please use nice latex and bibtex entries \cite{latex}. Do not spend time on formating your thesis, but on its content. 
 
\section{Motivation}
As we progress from livelihood fisheries to aquaculture industries, the global production and demand of fishes has drastically increased over several decades. According to \todo {add literature}

\todo { check this praragraph}
Underwater stereo-video measurement systems are used widely for counting
and measuring fish in aquaculture, fisheries and conservation management.
To determine population counts, aptial or temporal frequecies, most commonly using a point and click process by human operator.
Current research aims to use stereo 3d depth vision system. 
A fully automated process will requiere the detection and identification
of candidates for measerument, followed by the SNOUT to fork length measurement, as well as the counting and tracking of fish. This Thesis present a review and implemtation of techniques uses for the detection, Identifaction, measurement, counting and tracking of fish in underwater image sequences, including considreation of the changing in body shape.
The review will analyse the most commonly used approaches, leading to an evaluation of techniques most likely to be a  general solution to the complete processof detection, measurement, counting and tracking


\section{Problem Statement}
There is no need for a latex introduction since there is plenty of literature out there.

\todo { adding some text for structure this}
The monitoring of fish for stock assessment in aquaculture,
commercail fisheries and in the assessment of the effectiveness of biodiversity management strategies such as Marine Protected Areas and closed area management is essential for the economic and enviromental management of fish population,
Video based techniques for fishery independent and non-destructive sampling now widely accepted.
The advantages od suing stereo-video for counting the numbers of fish. measuring their lenght and defining the sample area have been well demostrated.
However, the time lag and cost of processing video imagery decreases the cost eo miffectiveness and uptake of this technology. Current research aims t



\section{Next Section}
There is no need for a latex introduction since there is plenty of literature out there.
