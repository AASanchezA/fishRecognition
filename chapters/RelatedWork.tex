\chapter{Related Work}
\label{chapter:Related Work}



% Here starts the thesis with an introduction. Please use nice latex and bibtex entries \cite{latex}. Do not spend time on formating your thesis, but on its content. 
 
\section{Related Work}
This thesis focuses on detecting fishes; therefore, this chapter concentrates on
works that deal with object detection, specifically in underwater application.
In \citet{Shortis2013} reviewed the state of arts on the field of identification 
and measurement of fish, using underwater stereo-video sequences. The most interesting 
approaches describes in this paper, are related with automated measurement of fish 
in video sequences as describe by \citet{Tillett2000} in what is one of the first 
published reports on successful, operational, automated measurement system. The 
technique is based on 3D Point Distribution (PDM), which are composed of landmark 
locations on the outline of the fish, in this case Atlantic Salmon held in a small 
aquaculture tank. The PDM specific to the species is developed from a small sample 
of fish defined by manual measurement of stereo-images, leading to a mean shape 
and an estimate of the variation based on principal component analysis. The PDM 
is independent of the scale and orientation, but is limited only to the silhouette 
of the fish and does not model the full body shape.
This work, also analyse the methodologies for detection of fishes which comprises
two steps: Identification and subsequent delineation of the fish outline. Most of the
existing work on fish detection employ either the differences between successive 
images \citet{Spampinato2008,Tillett2000} or histogram-thresholds to segment a 
varying number of candidate regions in the frames. Active contours( also called snake)
are especially useful for delineating objects like fish bodies that are difficult to 
model with rigid geometric primitives. Moreover, active contours can be independent 
from edge gradients with flexibility in initialisation \citet{Chan2001}. 
The area-based active contour model \citet{Chan2001} is based on the on techniques 
of curve evolution and level sets. While parametric active contours cannot handle 
automatic change of topology, level sets allow for splitting 
and merging in a natural way and are thus more suited for detection of an unknown
number of fish in a video image. 
In \citet{Shortis2013}, they also tackle the different technique apply in measurement.
Underwater stereo systems are widely used to capture video of swimming fish for 
subsequent measurement. the simplest form of measurement is the fish snout to tail 
length which can be calculated if these two points can be identified in the stereo 
pair of images. in our days this is done manually in most cases, with a favourable 
orientation of the fish to the cameras and a multiple measurements within in the 
sequences of frames do improve the precision of the measurements. One important 
point to be noted is that fish are deformable and the euclidean distance from snout to tail
changes as the fish swims.
Template matching is one of the primitive methods that can be employed to accurately 
locate fish snout and tail in video frames. First, individual templates (usually 
rectangular image regions) centered on the snout and tail mid-points are extracted 
from sample videos. Then an efficient template matching strategy is employed to 
locate these templates in target videos. A certain degree of robustness against 
illumination changes can be achieved by using correlation between template and 
image regions of interest, instead of taking their absolute differences \citet{Mahmood2012}.
However, template based methods fail in the presence of perspective or affine transformations, 
requiring either use of multiple templates that capture appearance variations from 
different viewing angles, or using more sophisticated matching techniques that are 
invariant to affine or perspective transformations. 
These enhancements also significantly increase computational complexity of the template matching step.
A better way of locating snout and tail is to use Haar-like features in a boosted 
classifier setup \citet{Viola2001} that has shown high object detection accuracy, 
besides being able to operate in real-time. The method is in wide use for face detection.
To train the classifier, manually cropped images of the target object (snout or tail) are used so that the classifier can learn which features (among a set of possibly thousands of features) can locate the target with high accuracy. These features, once learned, are then used to construct the object classifier that can locate the presence of the object in cluttered scenes. Due to their high detection speed and ability to perform a scale-space search, Haar classifiers are a promising candidate for locating snout and tail of fish in underwater images. The results of independent detection of the snout and tail using Haar detectors can be further improved using relationships between the detected snouts and tails, for instance by constraining the search for tail detection based on the results of snout detection and vice versa.

Furthermore, since we are detecting fishes, it is safe to limit our scope to articulated
object detection. This lead to the idea that our problem is similar to human pose
estimation. and specifically to 

\todo{ add ramanan and linemod approach}


% There is no need for a latex introduction since there is plenty of literature out there.


% \section{Next Section}
% There is no need for a latex introduction since there is plenty of literature out there.
