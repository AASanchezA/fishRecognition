\chapter{Methods}
\label{chapter:Methods}



% Here starts the thesis with an introduction. Please use nice latex and bibtex entries \cite{latex}. Do not spend time on formating your thesis, but on its content. 
 
\section{Theorical Background}
There is no need for a latex introduction since there is plenty of literature out there.
\todo{ intro to approach}
 

\section{Notation and Symbols}
This section introduces the common mathematical notations and symbols used in this 
chapter.
Normal formatting such as $a$ and $A$ are used to indicate integers or real numbers while letters in scripts such $\mathcal{A}$ are reserved for sets. In addition, 
we use the symbols $\mathbb{R}$ for real numbers.
For 3D vectors, we use the uppercase bold letters $\mathbf{A}$ while for 2D vectors, 
we use the lowercase bold letters $\mathbf{a}$. If the vector if homogeneous, it will
be explicitly define; otherwise, the vector is assumed to be inhomogeneous. A vector
from point $A$ to $B$ is presented as $\overrightarrow{AB}$. Furthermore, matrices use
monospace font such as $\mathtt{A}$. If the dimension are specified such as $\mathtt{A}$$_{mxn}$,
$m$ would indicate the number of rows while $n$ indicates the numbers of columns.
Regardings accents, a hat on a vector such as $\hat{a}$ or $\hat{A}$ indicates the normalized
unit vector which means that $\hat{a}=\frac{a}{\|a\|}$ or $\hat{A}=\frac{A}{\|A\|}$.
A tilde on a 3D vector indicates that the last coordinate is removed; thus, the vector 
$A=(x,y,z)^T$ have $\hat{A}=(x,y)^T$  which is the projection of $A$ in the $xy$-axis,
while the vector with homogeneous coordinate $B=(x,y,z,1)^T$ have $\hat{B}=(x,y,z)^T$ 
which is the inhomogeneous coordinate of $B$. Lastly, a dot on top of variable indicates
the converged value of the variable after an algorithm is performed. For instances,
after using mean shift on $a_i$, it converges to a value $\dot{a}_i$

\section{Part Based Model}
There is no need for a latex introduction since there is plenty of literature out there.

\section{Linemod}
There is no need for a latex introduction since there is plenty of literature out there.
